\documentclass{article}
\usepackage{amsmath, amsfonts, amsthm, amssymb}  

\usepackage{secdot}
\usepackage{textcomp}
\usepackage{epsfig}
\usepackage{cprotect}
\usepackage[T1]{fontenc}
\usepackage{epstopdf}
\usepackage{hyperref}
\usepackage{rotating}
\usepackage{graphicx}
\usepackage{caption}
\usepackage{subcaption}
\usepackage{multirow}
\usepackage{setspace}
\usepackage{array}
\usepackage{fancyhdr}
\usepackage{lastpage}
\usepackage[T1]{fontenc}

\usepackage{geometry}
\geometry{letterpaper, left=1in, right=1in, top=1in, bottom=1in}

\pagestyle{fancy}
\fancyhf{}
\rhead{\thepage/\pageref{LastPage}}
\lhead{OSU ECEN 2233 - Logic Design - Fall 2023}
\rfoot{\LaTeX}


% ----- Identifying Information -----------------------------------------------
\newcommand{\myassignment}{Lab 4: Using IP and Ethernet}
\newcommand{\myduedate}{Assigned: Monday 10/23; Due \textbf{Monday 11/6} (midnight)}
\newcommand{\myinstructor}{Instructor: James E. Stine, Jr.}
% -----------------------------------------------------------------------------

\begin{document}
\begin{center}
  {\huge \myassignment} \\
  {\large \myduedate} \\
  \begin{flushright}
  \myinstructor \\
  \end{flushright}
\end{center}

\section{Introduction}

Ethernet was developed at Xerox PARC between 1970s as a
method to allow Alto computers to communicate with each other. It
was inspired by the ALOHAnet, which Robert Metcalfe had studied as part of
his PhD dissertation and was originally called the Alto Aloha
Network.  Just like many engineering advancements, Robert Metcalfe and
his colleagues were in the right place at the right time in addition
to trying an idea that is practical and easy to implement.  I guess
the adage is the rest is history.
A really great YouTube video documenting the ideas of Ethernet can be
found here~\url{https://youtu.be/g5MezxMcRmk?si=UomwAQHmMgY6cvh4}.  

Personally, I think the idea is quite efficient and straight forward,
but it is so vital to today's society that I firmly believe its
something you should know about.  This laboratory will be about using
Ethernet from some pre-built Hardware Descriptive Language (HDL) or
sometimes called Intellectual Property (IP).  In many digital areas,
IP is essential to many products both commercially and in research
that you have to just add it to your design and try it.  This
laboratory will be about that very idea -- that is, to use specific IP
and understand how to use it.

\subsection{Ethernet Packets}


\bibliographystyle{IEEEbib}
\bibliography{lab4}

\end{document}
