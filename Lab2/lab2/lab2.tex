\documentclass{article}
\usepackage{amsmath, amsfonts, amsthm, amssymb}  

\usepackage{algorithm}
\usepackage{algpseudocode}
\usepackage{ascii}
\usepackage{secdot}
\usepackage{epsfig}
\usepackage{cprotect}
\usepackage[T1]{fontenc}
\usepackage{epstopdf}
\usepackage{hyperref}
\usepackage{hyperref}
\usepackage{rotating}
\usepackage{graphicx}
\usepackage{caption}
\usepackage{subcaption}
\usepackage{multirow}
\usepackage{amsmath}
\usepackage{setspace}
\usepackage{array}
\usepackage{fancyhdr}
\usepackage{fancyvrb}
\usepackage{lastpage}
\usepackage[T1]{fontenc}

\usepackage{geometry}
\geometry{letterpaper, left=1in, right=1in, top=1in, bottom=1in}

\pagestyle{fancy}
\fancyhf{}
\rhead{\thepage/\pageref{LastPage}}
\lhead{OSU ECEN 2233 - Logic Design - Fall 2023}
\rfoot{\LaTeX}


% ----- Identifying Information -----------------------------------------------
\newcommand{\myassignment}{Lab 2: Complex Combinational Logic and Debugging : Hardware-based Secure Hash Algorithm}
\newcommand{\myduedate}{Assigned: Monday 9/25; Due \textbf{Monday 10/16} (midnight)}
\newcommand{\myinstructor}{Instructor: James E. Stine, Jr.}
% -----------------------------------------------------------------------------

\begin{document}
\begin{center}
  {\huge \myassignment} \\
  {\large \myduedate} \\
  \begin{flushright}
  \myinstructor \\
  \end{flushright}
\end{center}

\section{Introduction}

Digital systems are important in all areas of society and using
combinational logic is a key element to this
development~\cite{ddca-riscv}.  This
laboratory will give you more experience with combinational logic
for digital systems.  
Security is a major design concern for all devices, including those  we
use every day, such as cellular phones and computers.
This laboratory will deal with a security cipher that was important in
the 1990s.  However, this security encryption standard, called Data
Encryption Standard (DES)~\cite{fips463, Biryukov2005}, fell out of
favor because we
could use
digital logic to help break into these devices.

For this laboratory, we are going to develop a hardware-based Secure Hash Algorith (SHA)
implementation
in two parts.  The primary part of this laboratory will involve designing the 
SHA algorithm found in this laboratory.
Security is not only important but many people feel that its one of the most important
topics that engineers
need to learn in the 21st century.  Therefore, I
believe this laboratory will be a great experience in learning some
security and the basics related to making sure someone does not have
unwanted guests within their systems.  The ideas can also be
translated easily into more advanced cryptographic systems, such as
Advanced Encryption Standard (AES) function that is
commonly used in bitcoin and web-based authentication.

The most widely used cryptographic operations are encryption and
decryption for secrecy, hashing for integrity, and signatures for
authenticity. Pure software implementations are slow, power-hungry,
and vulnerable to timing attacks that can be exploited
remotely. Modern instruction sets provide dedicated cryptography
instructions that are faster, simpler, and provide better performance
than pure software implementations. Moreover, having cryptographic
instructions promotes standardized software and reduced code size,
which helps reduce the risk of inadvertent security flaws. 

The Secure Hash Algorithm 2 (SHA-2) is the hash function used in most
internet protocols, such as TLS, SSL, PGP, etc., as well as to verify
transactions in Bitcoin and other cryptocurrencies. It was designed by
the US National Security Agency (NSA) and was first published in
2001. It is now a standard maintained by the National Institute of
Standards and Technology~\cite{1250396}.
SHA-2 generates $224$-, $256$-, $384$-, or $512$-bit message digests,
replacing the SHA-1, MD4, and MD5 algorithms that produced shorter
digests and are no longer considered secure. 
Other flavors of SHA-2 are similar but truncate the digest to
fewer bits after it is computed, trading compactness for reduced
security. 

\subsection{Security Basics}

Cryptography is the science of hiding the meaning of
messages. Although it has gained interest in recent decades for
computer security, it has been around at least since Julius Caeser
wrote B in place of A to prevent the Gauls from reading his messages
to his generals.
For our field, Claude Shannon (1916-2001) originally thought of
applying these ideas related to software and hardare in terms
of their confusion and diffusion~\cite{shannon1} and later expanded this
into his communication theory of secrecy systems~\cite{6769090}.
Cryptography uses many primatives, including symmetric ciphers,
asymmetric (also called public key) ciphers, hash functions, and
cryptographic protocols.

Encryption security can be broken down into the basic idea of using a
password
or a key to grant access to information.  The message that we want to
encrypt is
known as the \textit{plaintext} and the resulting encrypted message is
known as the
\textit{ciphertext}.
Symmetric ciphers use the same secret key for both encryption and
decryption. It is an efficient way to encrypt bulk data.
These symmetric-key
algorithms
also benefit from straightforward decryption operations: decryption is either
the exact same as encryption or all the steps from encryption simply performed
in reverse-order.  Figure~\ref{crypto_hw.png}
shows a symmetric-key encryption system
that encrypts a plaintext message with a key to produce the
ciphertext. In a good system, deducing the plaintext from the
ciphertext without the key is impractically difficult.
\begin{figure}
  \centering
  \includegraphics[scale=1.0]{crypto_hw.png}
  \caption{Basic Symmetric Cryptographic Hardware Block Diagram}
  \label{crypto_hw.png}
\end{figure}

\subsection{Rotation}
\label{swizzle.sec}

One of the most common operations in cryptography is called rotation.
Rotation is similar to shifting except anything that is shifted out of
a block gets put back into the block on the other side.  In other
words, a rotation or sometimes called a circular shift is an operation
similar to shift except that the bits that fall off at one end are put
back to the other end.  It is easy to see this as an example.

If we have $n$ that is stored using $8$ bits.
A left rotation of \verb!n = 1110_0101! by $3$ makes
\verb!n = 0010_1111! (Left shifted by 3 and first 3 bits are put back
in least-significant positions.  Fortunately, SystemVerilog (SV) makes
rotation and shifting easy to create with bit-swizzling.

Bit swizzling is SV is achieved with the curly braces \{ and \}.
Using an example from our textbook~\cite{ddca-riscv}, where $y$ is
given as a $9$-bit value
$c_2c_1d_0d_0d_0c_0101$ using bit swizzling operations.  This can be
created in SV by the following statement.
\begin{verbatim}
assign y = {c[2:1], {3{d[0]}}, c[0], 3’b101};
\end{verbatim}
In reality, the \{\} operator is used to concatenate busses. The
\verb!{3{d[0]}}! indicates three copies of \verb!d[0]!.
As stated in our textbook do not confuse the $3$-bit binary constant
\verb!3‘b101! with a bus named $b$.
It is important to note that it is critical to specify the length of
$3$ bits in the constant; otherwise, it would have had an unknown
number of leading zeros that might appear in the middle of $y$.
If $y$ were wider than $9$ bits, zeros would be placed in the most
significant bits.

\section{Hash Functions}

Cryptographic hash functions are important elements of
cryptography. They transform a variable-length message into a short
numerical fingerprint called a message digest. Hashes are used to
verify data integrity and as a building block for digital
signatures. Any alteration to a message will corrupt the message
digest. Passwords are usually stored in hashed form instead of
plaintext, so stealing the password file will not reveal the password
itself. A good hash function has several properties:
\begin{itemize}
\item Avalanche Effect - Any change in the message will, with very high probability, change 
  many bits of the digest
\item Pre-Image Resistant - Given a message digest, it is
  computationally infeasible to find the original message.
\item Collision Resistant - Given a message digest, it is
  computationally infeasible to find
  another message that produces the same digest
\item Fast and easy to compute
\end{itemize}

One of the most commmon uses of hash functions are in use with our use
of the GitHub repository which is a SHA-1 digest.
The hash function is used to differentiate
which modification is made for a given repository.  This can be seen
in Figure~\ref{sha.fig}.
But to make these hashes or ids easier to handle it also supports using a
short version of the id. The short commit id can actually be any
number of characters as long as it's unique for a commit within the
same repo.
To conserve space, GitHub actually shortens
the hash even though its $40$ characters or $160$-bits in length (i.e.,
\verb!38fefbbd46d62f394949b0448707c4f24cb60a3a!). 
\begin{figure}
  \centering
  \includegraphics[scale=0.5]{github.png}
  \caption{Example GitHub repository showing Hash of \texttt{0x38fefbb}}
  \label{sha.fig}
\end{figure}

For this laboratory, we will implement SHA-$256$ which is the most
popular form that people are most familiar with. Most Linux
distributions come with programs to compute the SHA-256 hash function
to verify data integrity. For example, the following produces the hash
for “Hello World!” using SHA-256: echo -n "Hello World!" |
sha256sum. If somebody changed the exclamation point to a question
mark, sha256sum would give a different hash, revealing that the
message had been corrupted.
\begin{verbatim}
7f83b1657ff1fc53b92dc18148a1d65dfc2d4b1fa3d677284addd200126d9069  -
\end{verbatim}
You can also interactively watch SHA-$256$ run at
\url{https://sha256algorithm.com}.

Table summarizes the structure of SHA-$256$ and SHA-$512$. The hash
operates on a message $M$ comprising $N$ $m$-bit blocks; it is padded if
necessary to be an integral number of blocks. Each block is formed
from $16$ $w$-bit words. Word $j$ of block $i$ is denoted $M_j^{i}$.
The message digest (also called the hash) $H$ is formed from $8$ $w$-bit
words. Each block goes through $r$ rounds of hashing, which involve
applying some shifts, rotates, and logical and addition
operations. The hashing steps involve sigma ($\Sigma$/$\sigma$) functions that are
expressed in terms of right rotations (\verb!ror!) and right shifts (\verb!>>!) of
the words. The hash is initialized with $8$ $w$-bit constants $H_j^{0}$
and uses $r$ $w$-bit round constants $K_t$ tabulated in the SHA-2
specification.
\begin{table}
  \centering
  {\footnotesize
  \begin{tabular}{|c|c|c|c|c|c|c|} \hline
    SHA-2 Algorithm & Var & SHA-256 & SHA-512 \\ \hline \hline
    Msg Digest Size (bits/bytes) & d & 256/64 & 512/128 \\ \hline
    Block Size (bits/bytes) & m & 512/128 & 1024/256 \\ \hline
    Word Size (bits/bytes) & w & 32/4 & 64/16 \\ \hline
    Rounds & r & 64 & 80 \\ \hline
    $\Sigma_0^{d} (x)$ & &
    \verb!(x ror 2) ^ (x ror 13) ^ (x ror 22)! &
    \verb!(x ror 28) ^ (x ror 34) ^ (x ror 39)! \\ \hline
    $\Sigma_1^{d} (x)$ & &
    \verb!(x ror 6) ^ (x ror 11) ^ (x ror 25)! &
    \verb!(x ror 14) ^ (x ror 18) ^ (x ror 41)! \\ \hline
    $\sigma_0^{d} (x)$ & &
    \verb!(x ror 7) ^ (x ror 18) ^ (x >> 3)! &
    \verb!(x ror 1) ^ (x ror 8) ^ (x >> 7)! \\ \hline
    $\sigma_1^{d} (x)$ & &
    \verb!(x ror 17) ^ (x ror 19) ^ (x >> 10)! &
    \verb!(x ror 19) ^ (x ror 61) ^ (x >> 6)! \\ \hline
  \end{tabular}
  }
  \caption{SHA-2 structure and sigma operations}
\end{table}

The Secure Hash Algorithm~\cite{1250396} is one of the most widely
utilized message digest functions.  We will be implementing this
hardware cryptographic system in hardware on the FPGA.  It will
involve two basic steps:
\begin{enumerate}
  \item Proprocessing - This sounds exactly what it sounds like; that
    is, the message $x$ has to be padded to fit a size of a multiple
    of $512$ bits.
    \item Hash Computation - Each message block xi is processed in
      four stages with $64$ rounds each as shown in
      Figure~\ref{sha2.fig}.  For this laboratory, we will handle this
      combinationally.  Later in the semester, we will explore ways we
      can do this with sequential logic too.
\end{enumerate}
\begin{figure} [t!]
  \centering
  \includegraphics[scale=0.7]{sha256.png}
  \caption{SHA-256 Block Diagram}
  \label{sha2.fig}
\end{figure}

\subsection{ASCII}

ASCII stands for American Standard Code for Information
Interchange. Computers can only understand numbers, so an ASCII code
is the numerical representation of a character such as 'a' or '@' or
an action of some sort. ASCII was developed a long time ago and now
the non-printing characters are rarely used for their original
purpose. Table~\ref{ascii.tbl} is an abbreivated
ASCII character table. ASCII was
actually designed for use with teletypes, so the descriptions are
somewhat obscure.
ASCII has expanded for other uses, however, its use is still
utilized as it is easy to use for the english alphabet (sometimes being
called US-ASCII).
ASCII has just $128$ code points, of which only $95$ are printable
characters, which severely limit its scope.  However, this has been
expanded to handle other alphabets.  More elaborate ASCII tables can
be found here: \url{https://en.wikipedia.org/wiki/ASCII}.

You will need to convert your text that you want to create a hash for
with ASCII codes.  You can either look it up manually using
Table~\ref{ascii.tbl} or use the included Python file to convert it
automatically.  Use these hex values to include in your SV
manual that will generate the hash value.  
\begin{table} [t!]
  \centering
  {\small
  \begin{tabular}{|c|c|c||c|c|c||c|c|c||c|c|c||c|c|c||c|c|} \hline
    Dec & Hex & Char & Dec & Hex & Char & Dec & Hex & Char & Dec & Hex & Char & Dec & Hex & Char\\ \hline \hline
    048 & 0x30 & 0 & 064 & 0x40 & @ &  080 & 0x50 & P        & 096 & 0x60 & \textquoteleft & 112 & 0x70 & p        \\ \hline
    049 & 0x31 & 1 & 065 & 0x41 & A &  081 & 0x51 & Q        & 097 & 0x61 & a              & 113 & 0x71 & q        \\ \hline
    050 & 0x32 & 2 & 066 & 0x42 & B &  082 & 0x52 & R        & 098 & 0x62 & b              & 114 & 0x72 & r        \\ \hline
    051 & 0x33 & 3 & 067 & 0x43 & C &  083 & 0x53 & S        & 099 & 0x63 & c              & 115 & 0x73 & s        \\ \hline
    052 & 0x34 & 4 & 068 & 0x44 & D &  084 & 0x54 & T        & 100 & 0x64 & d              & 116 & 0x74 & t        \\ \hline
    053 & 0x35 & 5 & 069 & 0x45 & E &  085 & 0x55 & U        & 101 & 0x65 & e              & 117 & 0x75 & u        \\ \hline
    054 & 0x36 & 6 & 070 & 0x46 & F &  086 & 0x56 & V        & 102 & 0x66 & f              & 118 & 0x76 & v        \\ \hline
    055 & 0x37 & 7 & 071 & 0x47 & G &  087 & 0x57 & W        & 103 & 0x67 & g              & 119 & 0x77 & w        \\ \hline
    056 & 0x38 & 8 & 072 & 0x48 & H &  088 & 0x58 & X        & 104 & 0x68 & h              & 120 & 0x78 & x        \\ \hline
    057 & 0x39 & 9 & 073 & 0x49 & I &  089 & 0x59 & Y        & 105 & 0x69 & i              & 121 & 0x79 & y        \\ \hline
    058 & 0x3A & : & 074 & 0x4A & J &  090 & 0x5A & Z        & 106 & 0x6A & j              & 122 & 0x7A & z        \\ \hline
    059 & 0x3B & ; & 075 & 0x4B & K &  091 & 0x5B & [        & 107 & 0x6B & k              & 123 & 0x7B & \char`\{ \\ \hline
    060 & 0x3C & < & 076 & 0x4C & L &  092 & 0x5C & \char`   & 108 & 0x6C & l              & 124 & 0x7C & |        \\ \hline
    061 & 0x3D & = & 077 & 0x4D & M &  093 & 0x5D & ]        & 109 & 0x6D & m              & 125 & 0x7D & \char`\} \\ \hline
    062 & 0x3E & > & 078 & 0x4E & N &  094 & 0x5E & \^{}     & 110 & 0x6E & n              & 126 & 0x7E & \~{}     \\ \hline
    063 & 0x3F & ? & 079 & 0x4F & O &  095 & 0x5F & \char`\_ & 111 & 0x6F & o              & 127 & 0x7F & \DEL     \\ \hline
  \end{tabular}
  }
  \caption{Common English ASCII Characters}
  \label{ascii.tbl}
\end{table}


\subsection{Preprocessing}
\label{padding.sec}

Before the actual hash computation, the message $x$ has to be padded to
fit a size of a multiple of $512$ bit. For the internal processing, the
padded message must then be divided into blocks. This will be
dependent on the size of the message you plan on sending (e.g., SHA-256).

For example, Assume that we have a message $x$ with a length of $l$
bits. To obtain an overall message size of a multiple of $512$ bits, we
append a single “1” followed by $k$ zero bits and the binary $64$-bit
representation of $l$. Consequently, the number of required zeros $k$ is
given by
\begin{eqnarray*}
  k & = & 512 - 64 - 1 - l
\end{eqnarray*}

For example, the (8-bit ASCII) message “abc” has length $8 \times 3 = 24$,
so the message is padded with a one bit, then $448 - (24 + 1) = 423$
zero bits, and then
the message length (i.e., $64$), to become the $512$-bit padded
message.  This is illustrated in Figure~\ref{padding.fig}
\begin{figure}
  \centering
  \includegraphics[scale=0.4]{padding.png}
  \caption{Padding example for ``abc'' for a $512$-bit padded message~\cite{1250396}}
  \label{padding.fig}
\end{figure}

This can be done easily within SystemVerilog with bitswizzling as
indicated in Section~\ref{swizzle.sec}.
An example of this could be the following SV Hardware Descriptive
Language (HDL):
\begin{verbatim}
assign padded = {message, 1'b1, {zero_width{1'b0}},  {back_0_width{1'b0}}, MSG_SIZE};
\end{verbatim}


\subsection{Hash Computation}

The real advantage to using digital logic is that much of the
computation can be done in parallel.  That is, when one piece of logic
is being computed, another part can be done at the same time reducing
the total amount of time for the computation.  Software, on the other
hand, typically is slower as it requires waiting for previous
operations to complete before it can move forward.

\subsubsection{Modular Addition}
\label{modular.sec}

The computation of the hash function is a combination of rotations,
shifts, and additions.  Addition is tricky as it has to be done not to
exceed the size of the addition (in this case, since it is a block of
$32$-bits, then it should not exceed $2^{32}$ or modulo $32$.  This is
called modular arithmetic and can be summarized in the following equation.
\begin{eqnarray*}
  \mid X+Y \mid_m = \Biggl\{
   \begin{array}{@{}ll@{}}
      X+Y,      & \text{ if $X+Y < m$}  \\
      X+Y-m,    & \text{ if $X+Y \geq m$} 
   \end{array}
\end{eqnarray*}  
Modular arithmetic is key to many cryptographic algorithm.
Fortunately, for our implementation it is quite easy as we just have
to drop off the MSB.

\subsubsection{constants}

Before hash computation begins for each of the secure hash algorithms,
the initial hash value,
$H^0$, must be set. For SHA-256, the size and number of words in $H^0$
depends on the message digest size.  These words area obtained by
taking the first thirty-two bits of the fractional parts of the square
roots of the first eight prime numbers to ensure the numbers are
random.  That is, the fractional parts of $2, 3, 5, 7, 11, 13, 17,
19$.  These value are different for differt versions of the Secure
Hash Standard~\cite{1250396}.
\begin{eqnarray*}
  H_0^0 & = & \verb!0x6a09e667!\\
  H_1^0 & = & \verb!0xbb67ae85!\\
  H_2^0 & = & \verb!0x3c6ef372!\\
  H_3^0 & = & \verb!0xa54ff53a!\\
  H_4^0 & = & \verb!0x510e527f!\\
  H_5^0 & = & \verb!0x9b05688c!\\
  H_6^0 & = & \verb!0x1f83d9ab!\\
  H_7^0 & = & \verb!0x5be0cd19!\\
\end{eqnarray*}

There are some additional constants utilized within the main hash
computation.  These are computed similar based on the cube roots of
the first sixty-four prime numbers and labeled
$K_0^{256}, K_1^{256}, \ldots K_{63}^{256}$.
\begin{verbatim}
428a2f98 71374491 b5c0fbcf e9b5dba5 3956c25b 59f111f1 923f82a4 ab1c5ed5
d807aa98 12835b01 243185be 550c7dc3 72be5d74 80deb1fe 9bdc06a7 c19bf174
e49b69c1 efbe4786 0fc19dc6 240ca1cc 2de92c6f 4a7484aa 5cb0a9dc 76f988da
983e5152 a831c66d b00327c8 bf597fc7 c6e00bf3 d5a79147 06ca6351 14292967
27b70a85 2e1b2138 4d2c6dfc 53380d13 650a7354 766a0abb 81c2c92e 92722c85
a2bfe8a1 a81a664b c24b8b70 c76c51a3 d192e819 d6990624 f40e3585 106aa070
19a4c116 1e376c08 2748774c 34b0bcb5 391c0cb3 4ed8aa4a 5b9cca4f 682e6ff3
748f82ee 78a5636f 84c87814 8cc70208 90befffa a4506ceb bef9a3f7 c67178f2
\end{verbatim}

\subsubsection{Main SHA-256 computation}

SHA-256 in hardware is typically very easy if taken systematically.
For this laboratory, we will only use 1 $m$-bit block or 1 $512$-bit
block.  In theory, hardware may be composed of multiple blocks.

SHA-256 can be used to hash a message, $M$, having a length of $l$
bits, where $0 \leq l < 2^{64}$. The
algorithm uses 1) a message schedule of sixty-four $32$-bit words,
2) eight working variables of $32$
bits labeled $a$ through $h$, and 3) a hash value of eight $32$-bit words.
The final result of SHA-256 is a $256$-bit
message digest.
The words of the message schedule are labeled
$W_0, W_1, \ldots, W_{63}$. The
eight working variables are labeled $a$, $b$, $c$, $d$, $e$, $f$, $g$,
and $h$.
The words of the hash value are labeled $H_0^i, H_1^i, \ldots H_7^i$.
which will hold the initial hash value, $H^0$, replaced by each
successive intermediate hash value
(after each message block is processed), $H^i$, and ending with the
final hash value, $H^N$.  For cryptographic systems, this is sometimes
called a round and its similar to 

The operation looks more complicated than it is but its just a series
of computations in parallel.  The key is to get the order of
processing correct and, of course, the number of steps.
A key piece item to remember is that
addition (+) is performed modulo $2^{32}$ as described in Section~\ref{modular.sec}.
\begin{enumerate}
  \item Processsing: set the inital hash values $H^0$ as previously
    specified as well as the padding (see Section~\ref{padding.sec}).
  \item Prepare the message:
    \begin{itemize}
  \item Since we are only operating on $1$ group or $N=1$, we can
    break the computation down into blocks of $32$ for each part
    of the message and operate on the message for $64$ rounds (i.e.,
    $0 \leq t 15$).
    That is, Each message block, $M^1, M^2, \ldots, M^{64}$. 
    is processed in order.
  \item For blocks $16 \leq t \leq 63$, we need to compute
    $W_t = \sigma_1^{256} (W_{t-2}) + W_{t-7} + \sigma_0^{256} (W_{t-15}) + W_{t-16}$
    where
    \begin{eqnarray*}
      \sigma_0^{512}(x) = \text{ror}^{7}(x) \oplus \text{ror}^{18}(x) \oplus (x~\verb!>>!~3)    \\
      \sigma_1^{512}(x) = \text{ror}^{17}(x) \oplus \text{ror}^{19}(x) \oplus (x~\verb!>>!~10)    
    \end{eqnarray*}
  \item This can be summarized as follows:
    \begin{eqnarray*}
      W_t = \Biggl\{
      \begin{array}{@{}ll@{}}
        M_t^i,      & (0 \leq t \leq 15)  \\
        \sigma_1^{256} (W_{t-2}) + W_{t-7} + \sigma_0^{256} (W_{t-15}) +  W_{t-16},  & (16 \leq t \leq 63) 
      \end{array}
    \end{eqnarray*}  
    
    \end{itemize}
    
    \item Initialize the seven working variables with the (i-1)st
      hash:
      \begin{itemize}
      \item $a = H_0^{i-1}$
      \item $b = H_1^{i-1}$
      \item $c = H_2^{i-1}$
      \item $d = H_3^{i-1}$
      \item $e = H_4^{i-1}$
      \item $f = H_5^{i-1}$
      \item $g = H_6^{i-1}$
      \end{itemize}

    \item Compute the following items for $t = 0:63$ using      
      $T_1 = h + \Sigma_1^{256} (e) + \text{Choice}(e,f,g) + K_t^{63} + W_t$ and
      $T_2 = \Sigma_0^{256} (a) + \text{Majority}(a,b,c)$.  These 
      equations are broken down as for $\Sigma_1$, $\Sigma_0$
      and $\text{Choice(x,y,z)} = (x \cdot y) \oplus (\overline{x} \cdot z)$ and
      $\text{Majority(x,y,z)} = (x \cdot y) \oplus (x \cdot z) \oplus (y
      \cdot z)$.  The values of $\Sigma_1^{256}$ and $\Sigma_0^{256}$
      are similar to the lower case versions above or except that it
      is only composed of xor and ror (i.e., no right shifts) or the following.
          \begin{eqnarray*}
      \Sigma_0^{512}(x) = \text{ror}^{2}(x) \oplus \text{ror}^{13}(x) \oplus \text{ror}^{22}(x)    \\
      \Sigma_1^{512}(x) = \text{ror}^{6}(x) \oplus \text{ror}^{11}(x) \oplus \text{ror}^{25}(x)    
    \end{eqnarray*}
  \begin{itemize}
  \item $h = g$
  \item $g = f$
  \item $f = e$
  \item $e = d + T_1$
  \item $d = c$
  \item $c = b$
  \item $b = a$
  \item $a = T_1 + T_2$
  \end{itemize}
  
  \item Add the working variables above to the current hash variable
    to the current variables
    \begin{itemize}
    \item $H_0^1 = a + H_0^0$
    \item $H_1^1 = b + H_2^0$
    \item $H_2^1 = c + H_3^0$
    \item $H_3^1 = d + H_3^0$
    \item $H_4^1 = e + H_4^0$
    \item $H_5^1 = f + H_5^0$
    \item $H_6^1 = g + H_6^0$
    \item $H_7^1 = h + H_7^0$
    \end{itemize}

    \item Finally, concatenate or squish all the values in the
      previous step together forming a $512$-bit message (i.e., $32$
      hexadecimal digits).
      \begin{eqnarray*}
        H_0^1 || H_1^1 || H_2^1 || H_3^1 || H_4^1 || H_5^1 || H_6^1 || H_7^1
        \end{eqnarray*}
    
\end{enumerate}





\subsection{Power, Performance and Area (PPA)}

For this laboratory, we are going to analyze the design with better
PPA.  That is, you should analyze your design for Power, Performance
and Area.  As opposed to previous laboratories, this procedure that
will be documented here is more robust and gives better numbers that
you can use to assess whether your design is credible or not.  As with
any digital design, engineers use PPA to assess the level of
difficulty, challenge, and effort needed for a design.

To assess your PPA for this design, you should determine its PPA after
implementation.  This is because some of the PPA results (e.g.,
timing) are not adjusted properly until the Implementation phase.  The
Implementation phase typically places and routes the design onto the
FPGA by connecting all the logic blocks that we read about in the
article that we looked at in Lab 0~\cite{7086413}.

To obtain the PPA results, you first have to run through your design
making sure that it is implemented correctly.  Then, you need to add
the following reports after the route stage (i.e., during
Implementation):
\begin{enumerate}
\item \verb!report_utilization! : Area
\item \verb!report_timing! : Performance
\item \verb!report_power! : Power
\end{enumerate}
You can the reports you need by clicking on the reports tab, right
mouse clicking, and then adding the report you need, as shown in
Figure~\ref{reports1.png}.  Once you add the report, it is easiest to
re-run the implementation to get the report.  Clicking on the option
gets you specific report which you can save.
\begin{figure} [b!]
  \centering
  \includegraphics[scale=0.3, width=\columnwidth]{analysis.png}
  \caption{Reports Window within Xilinx Vivado}
  \label{reports1.png}
\end{figure}

  
\section{Tasks}

Most of the blocks and their operation
have been given to you to help you understand the
problem better.
For those that are interested in more about cryptography and how
hardware can impact the future, I encourage you to read more about it
through searching on the Internet as well as this great
reference~\cite{10.5555/1721909}.
One of the hard parts of any engineering problem is
to understand what is going on and making sure you are correct.
Therefore, digital designers rely heavily on getting good data to make
sure they are right.  Typically, this is done either on paper and
pencil or through software.

We will use software for this approach
and use a piece of software written in Java.
If you need
to install Java on your machine at home or laptop, go to
\url{https://www.oracle.com/java/technologies/downloads/#java16} and
download the appropriate version.
The main part of the encryption output looks like the following in
Figure~\ref{des.java} after
typing \verb!java DES!.  The plaintext and
key are inside the \verb!DES.java! program
and can be easily modified, however, I wrote a method in Java that
checks the
parity so make sure you have a good key.  For example, in round 1,
\verb!L_1 = 0x8C13_B66C!, \verb!R_1 = 0xF3EF_C169! and
\verb!K_1 = 0x2080_66A2_53BA!.  As seen by the output in
Figure~\ref{des.java}, the plaintext \verb!0x2579_DB86_6C0F_528C! with
a key of \verb!433E_4529_462A_4A62! produces the correct ciphertext of
\verb!ECB5_4739_A183_2EC5!.
This could be checked by taking the ciphertext and decrypting it
through the algorithm.  Since the algorithm is symmetric, it utilizes
the same procedure for encryption or decryption except that the keys
are reversed.
There are several DES calculators
available online through Google search if you wish to validate the
result this way, as well. 
\begin{figure}
\begin{verbatim}
Original plain Text:    2579DB866C0F528C
Key:                    433E4529462A4A62
IV (for CBC mode):      0000000000000000

Encryption:

After initial permutation: 5646B9278C13B66C
After splitting: L0=5646B927 R0=8C13B66C

Round 1	        8C13B66C F3EFC169 208066A253BA
Round 2	        F3EFC169 25DAF255 C0B6508F6DC2
Round 3	        25DAF255 1890CFBF 44D6422CC355
Round 4	        1890CFBF AFB98FA0 62D142D3C4C6
Round 5	        AFB98FA0 8F76DBD7 28C143CC8789
Round 6	        8F76DBD7 C176D0E5 21411B9A764D
Round 7	        C176D0E5 C7401A8C 2501917AD3A0
Round 8	        C7401A8C B748825A 170891906D2B
Round 9	        B748825A 61239171 084949255DD5
Round 10        61239171 FE28B577 01690D8B80F3
Round 11        FE28B577 CDB650DE 012D81C7CF05
Round 12        CDB650DE 8B8270E5 512CA11A07DC
Round 13        8B8270E5 DDDBEE19 D1A480D9D185
Round 14        DDDBEE19 5F82D63F 5086864266A9
Round 15        5F82D63F B35B4964 709006FA390D
Round 16        B35B4964 850AC7BE C03E202F8437

Cipher Text: ECB54739A1832EC5
\end{verbatim}
\caption{Java output for encryption from DES.java}
\label{des.java}
\end{figure}

Verification is extremely difficult
because there are so many moving parts.  Use the
Java program to verify each block out of the HDL.  Although the Java
works based on bytecodes that are interpreted, I have found that some
machines have problems reading the Java bytecodes.  I am still not
quite sure why this is the case, however, there is an easy fix.
Therefore, I included a
Makefile that I wrote that allows you to compile the Java correctly.
Please type the following if you are having problems running the
code.  To run the tool, type \verb!java DES! at the command prompt.
\begin{verbatim}
make clean
make
\end{verbatim}
If you cannot run \verb!make! on your Windows box, just type the
\verb!javac! commands found within the Makefile on each Java file
(i.e., \verb!javac -d . -classpath . DES.java!).  

The main tasks for this laboratory
will be the following elements:
\begin{enumerate}
  \item Design the DES combinatianal block for both encryption and
    decryption in SystemVerilog and simulate with ModelSim.
  \item Use the Java verification tool to help you with verifying the
    correct operation within ModelSim.  There is also a decent online
    DES calculator available at \url{https://emvlab.org/descalc/} that
    shows a simplified input/output value from either encryption or decryption.  
  \item Implement a switch that indicates ECB or CBC modes and
    processes everything accordingly.
  \item Test at least $10$ random messages (i.e., plaintext) using
    $2$ random keys for both encryption and decryption.
  \item After verifying your design with a testbench in ModelSim,
    implement your design on the DSDB board and use the    
    $7$-segment display to display your plaintext and ciphertext.
    Since you only have four $7$-segment displays, you will not be
    able to show the entire plaintext, ciphertext or key, so you will
    have to figure a way to verify the operation.
  \item You should also design an option that displays a LED if the
      key is correct (i.e., that parity is correct or that it is
      odd).  Your hardware should work regardless of parity because it
      does not use these bits when computing the subkeys, but a LED
      should be lit up if the key is bad - i.e., it does not have odd
      parity.  The java code comes with a method to help check whether
      the parity is odd to help you validate the parity.
  \item Use the push buttons, switches, and LEDs to help you input
    your plaintext as well as debug operation and prove that your
    design works on your DSDB board.
    \item You should also analzye the PPA impact on your design. 
\end{enumerate}
This laboratory should involve \textbf{only combinational logic} and be
straight forward in creating Boolean logic with SystemVerilog.
Again, there are many parts to this design and based on experience, I
believe it will be easier to debug the key generation first and then
once this works, debug the encryption/decryption next.  The key
generation is slightly easier than operations like the Feistel block,
so it will optimize your design process if you focus on this block
first.  However, I would use the strategy that works the best for you.

\subsection{Testing and Stubbing Code}

You should use the testbenches you utilized for Lab0 and Lab1 to help
you test your design.  The design is completely combinational and
should not be any different in terms of structure than both of these
labs.  To get full credit, you should demonstrate that your design
works for both encryption and decryption by testing at least $10$
plaintext messages using at least $2$ different keys.  This is
basically testing $20$ vectors - the more vectors tested and the
methodology you use could possibly
earn you extra credit on this laboratory.

I have also given you some freebies to help you with this lab.  When
writing HDL or software, it is sometimes useful to \textit{stub} your
code.  A stubbed piece of code is a blank piece of software that has
most of your functions you believe will work for your design.
Fortunately, I have stubbed out your SV for you and you can use this
as a guide.  I also put some comments in the SV to help you know where
to instantiate certain items.
Inside the SV, I have also included the complete S-boxes
which are the substitution boxes you will use for
this laboratory.  All of the S-boxes work by giving them $6$-bits and
they produce $4$-bits as an output, as indicated previously.

I have also utilized a more advanced testbench that reads your
key, plaintext, and ciphertext from a file.  These are included in the
\verb!des.tv! files and $4$ examples are given.  The testbench should
read in the values on each edge of the clock as in Lab 1.  Although
this testbench outputs data to a file, you will find more information
can be found through debugging in ModelSim as documented in the next
subsection.

\subsection{Getting to know ModelSim and Debugging more in depth}

ModelSim is a professional Hardware Descriptive Language tool for
simulation and verification.  It has many neat features to help you
with debugging.  Although testbenches are the main vehicle for
understanding how to test a digital system, using ModelSim can save
you hours and days in debugging a design.  Therefore, we are also
going to introduce some new features of ModelSim that you should use
to help you with this laboratory. I also encourage you to use the
testbench skills you learned from Lab 1.

The features you will use in ModelSim are the \textit{Sim} and
\textit{Objects} window.  Normally both of these windows are present
when running a DO file, however, sometimes I find that they do not
open properly.  You may need to activate them in the View menu at the
top of ModelSim.  They should look like Figure~\ref{modelsim.png} when
activated.  Both of these windows are utilized with the Wave window.
\begin{figure} [t!]
  \centering
  \includegraphics[scale=0.3]{modelsim.png}
  \caption{ModelSim Sim and Objects Window}
  \label{modelsim.png}
\end{figure}

To use the two windows effectively, you should use the \textit{Wave}
to see the data at a certain time.  First, move your cursor to the
time you wish to investigate something - you should see a yellow line
indicating the time you are observing the data.  Next, you
should navigate to the
hierarchy of the module you wish to verify in the \textit{Sim} window
and the \textit{Objects} window will display all signals and values
that for that instance at a given time.  You might need to play around
with using these two windows together with the \textit{Wave} window,
but once you do you will find that its easy to debug what each block
is producing at a given time.

The "sim" window (orange) contains the hierarchy of the design.  The
top level shows the test bench (tb) with a expandable button to the
left.
By clicking the "+" it opens the hierarchy for all modules
instantiated in tb.  Clicking on the name of the instance changes
which objects (blue) are visible in the "objects" window.  You can
also add an
object to the wave by right clicking on the name of the object in the
"objects" window "Add Wave".  Your testbench and modules may use
different names but the same process applies to add signals to the
wave (purple).

You can save the wave by clicking in the wave window then clicking the
brown colored floppy disk icon in the toolbar.  (Third icon from the
left)  The saved file only contains the configuration of the wave not
the actual data. This allows you to recall the wave if you restart
modelsim at a later time.  To recall the wave you can type "do <name
of wave file> in the transcript (yellow).  You can also add this to
the do file so it always pulls up your wave every time the simulation
is run.

Modelsim has many extra features which can greatly aid in your debugging.
First let's discuss some tips and tricks.
\begin{itemize}
\item  If the toolbar gets disorderly, right click in the toolbar and
  select reset.
\item Signals in the wave by default show the full path name.  This can
  be changed to just the lowest level of hierarchy by clicking the
  ``toggle leafs name'' button in the lower left of the wave shown in
  Figure~\ref{modelsim-tips1.png}  
\item Zoom buttons are confusing.  The ``+'' zoom in is mostly useless.
  Use the yellow upside down ``T'' with magnifying glass to zoom in
  at the cursor, as shown in Figure~\ref{modelsim-tips2.png} in red.
\item The ``-'' zoom button works as expected.
\item If you select a signal in the wave viewer, ``Tab'' and ``Shift + Tab''
  will move the cursor to the next transition.
\item A multibit bus can be search for a specific value using the ``Search''
  buttons in the toolbar.  The blue left and right allows to the right of
  the ``Search'' button will search backwards (left) or forwards (right)
  in time, as shown in Figure~\ref{modelsim-tips2.png} in green.
\end{itemize}

At the risk of complicating things the ``data flow'' window can be very
helpful when debugging red X's.  Either in the objects window or the wave
window right click a signal and select ``Add to dataflow''. This opens
a new window where you can right click and select ``ChaseX'' or ``TraceX''.
These allow you to quickly find the source of an X.  If this does not make
sense you can skip.
\begin{figure} [t!]
  \centering
  \includegraphics[scale=0.4]{modelsim-tips1.png}
  \caption{ModelSim toggle leafs. In the red box, the left-most box is the ``Now'' row.}
  \label{modelsim-tips1.png}
\end{figure}
\begin{figure} [t!]
  \centering
  \includegraphics[scale=0.32]{modelsim-tips2.png}
  \caption{Search inside the green box and zoom controls in the red box.}
  \label{modelsim-tips2.png}
\end{figure}

\subsection{Extra Credit}

If you get done early, you can attempt some extra credit.  However, I
would only try this option if you get everything verified within your
design.  
One possible improvement is to work on optimizing the verification
of your design.  You can do any other modification (e.g., re-writing
the Java code) or the DES implementation in Java, as well.
%Yet another piece of extra credit is analyzing how fast you can run
%DES through the FPGA.  This will involve using Vivado to analyze how fast your
%design can be and doing some calculations on how fast you can encrypt
%and decrypt your data.



\section{Submission}

You should electronically hand in your HDL (all files that you want
us to see) into Canvas.
You should also take a printout of your waveform 
from your ModelSim simulation.  
Only one of your team members should upload
the files and/or lab report. Please contact
James Stine
(james.stine@okstate.edu) 
for more help.  Your
code should be
readable and well-documented. In addition, please turn in additional
test cases or any other added item that you used. 
Please also remember to document everything in your Lab Report using
the information found in the Grading Rubric.


    
\bibliographystyle{IEEEbib}
\bibliography{lab2}

\end{document}
